
% Третья глава работы 
\chapter{Название главы}
\label{chap3}

\section{Название секции}
\label{chap3:sec1}

% Пример оформления листинга
% 
Для этого на сервере был запущен виртуальный сервер 
\verb|xserv|, с IP-адресом \verb|10.130.64.15|:

\begin{verbatim}
  vzctl create 3006  --os template gentoo-x86
  vzctl set 3006  --name /xserv  --save
  vzctl set 3006  --nameserver 10.130.64.15
  vzctl start 3006
  vzctl enter 3006
\end{verbatim}
Запускаем ssh:
\begin{verbatim}
  /etc/init.d/sshd start 
\end{verbatim}
Добавим запуск демона ssh по умолчанию:
\begin{verbatim}
  rc-update add sshd default
\end{verbatim}

Далее запускаем NX-сервер:
\begin{verbatim}
  nxserver --start
\end{verbatim}
Если все в порядке, появляется сообщение:
\begin{verbatim}
  NX> 100 NXSERVER~--- Version 1.4.0-44 OS (GPL)
        NX> 122 Service started
        NX> 999 Bye
\end{verbatim}

 %%

\section{Название секции}
\label{chap3:sec2}

Текст.

\section{Название секции}
\label{chap3:sec3}

Текст.

%%% Local Variables:
%%% mode: latex
%%% coding: utf-8-unix
%%% TeX-master: "../default"
%%% End: